% approach.tex
\begin{par}
    \par \hspace{15pt} Inspired by the related reading materials, the group has decided that the generative networks of the project's model should also be a temporal model, which it would have information related to both the entire track and the local segments of the music for pattern recognition. A coarse architecture of the model is demonstrated in Figure \ref{fig:archie}. 
    
    \subsection{Pre-Generator Ablations} % (fold)
    \label{sub:Pre-Generator Ablations}
    \par \hspace{15pt}  First, the completed tracks of other instruments are collected and feed into a classification/encoding algorithm. There are many different music styles such as Jazz, Blues and Rock. Each of these styles has its certain nuances, phrases, techniques, and sounds which are associated with the style of drumming that accompanies it. thus, music genre classification is critical. The algorithm would produce a drum genre space that will be utilized as the global inductive bias vector $z$ for the segmented generators. After reading some related materials upon music genre classifications \cite{transfer_classification, active_classification, cnn_classification, svm_classification}, the group proposed three methods for the classification/encoding algorithm to implement this part of the network.

        \subsubsection{CNN Encoders} % (fold)
        \label{ssub:Convolution Neural Network Encoders}
            \par \hspace{15pt} For the first option, the group would like to implement a conventional CNN encoder for the classification/encoding task. It is expected to be the most computationally intensive implementation but the most mature encoding structure available as of today. The group would probably inspect both the CNN encoder with freezed implementation during generation task or continuously fine-tuning the CNN encoder while training for the generation task.

        % subsubsection Convolution Neural Network Encoders (end)

        \subsubsection{Subspace Learning} % (fold)
        \label{ssub:Subspace Learning}
            \par \hspace{15pt}  For the second option, the group would like to test the classification task with the subspaces learning technique. Subspace learning should be much faster in inference comparing to the CNN encoder. \begin{par}
    % Because of the large scale of music data representation, doing training data Dimension Reduction (DR) and Principal Component Analysis(PCA) are essential to a lighter-load training process. Some related works use singular value decomposition(SVD) to find principal components to approximate the structure of the training data \cite{pca}. 
    
    For the music genre classification task of this project, the team currently plans to first group the training data into different music genres, and then use SVD to obtain an orthonormal basis for each genre space. Some related training can be seen in \cite{pca}. After having learned the genre subspaces, classification can be done by comparing the projections of one particular test data onto the orthogonal complement of the genre subspaces.
\end{par}
        % subsubsection Subspace Learning (end)

        \subsubsection{Perceiver} % (fold)
        \label{ssub:Perceiver}
            \par \hspace{15pt}  In the last option, one of the student in the group would like to implement an attention-based model, namely the Perceiver \cite{perceiver}. Since the output space of the perceiver model would be a logits vector, it can also serve as a traditional classifier/encoder structure. This part of the project would potentially be adopted from one of the student's EECS-542, Advanced Computer Vision course project. 
        % subsubsection Perceiver (end)
    % subsection Pre-Generator Ablations (end)

    \subsection{Discriminator Ablations} % (fold)
    \label{sub:Discriminator Ablations}
        \par \hspace{15pt}  The group would also like to do a study upon the ablation of the discriminator architecture, which the group has two variations in mind. 

        \subsubsection{No Discriminator} % (fold)
        \label{ssub:No Discriminator}
            \par \hspace{15pt} The most natural and intuitive idea the group has raised and would like to test on is completely removing the discriminator architecture from the original MuseGAN model. Thus instead of using a discriminator, the group would like to see if the model would solely work with L2-norm loss between the generated and the labels.
            \usepackage{amsmath}
\newcommand\norm[1]{\left\lVert#1\right\rVert}
\begin{par}
    \par \hspace{15pt} To evaluate the similarity between the generated vector and 
    the vector in the data sample-i.e. to check whether the model-generated drum beats 
    fit with provided music notes-we propose (as an initial proposition) to use Euclidean distance for 
    measuring the similarity between two vectors. Specifically, the group is considering L1 or L2 
    distance metric. The error calculation works as the following: 
    \begin{equation}
        \emph{J}(w) = \sum_{i=1}^{N} \norm{G(\mathbf{z})-x^{(i)}}
        
    \end{equation}
    where $G(\mathbf{z})$ represents generated drum beat vector, and $x^{(i)}$
    represents the data sample in training data set. The discriminator will denote the generated drum 
    beat as adequate as long as the above error is lower than a threshold value.

\end{par}

        % subsubsection No Discriminator (end)

        \subsubsection{Traditional Discriminator} % (fold)
        \label{ssub:Traditional Discriminator}
            \par \hspace{15pt}  The second intuitive method that came to the group was to use a traditional discriminators in GANs \cite{generative} instead of the probabilistic Wasserstein distance discriminator in the MuseGAN model. 
        % subsubsection Traditional Discriminator (end)

    % subsection Discriminator Ablations (end)
    

\end{par}