\begin{par}
    \par The team proposes to use Generator Network to generate pieces of music. Each input of the Generator has 
    multiple sub-components, and each sub-component of the Generator has three channels-first channel is a randomly 
    extracted song segment, the second channel contains the genre vector of the song, and lastly the third channel contains the positional
    encoding of the notes in the song segment. All the song segments in each individual input are extracted from the same song, meaning
    they share the same genre vector, which, as the name indicates, tells the Generator the genre of the song that's being fed in; in order
    to preserve a richer spatial connectivity, the team decides to add positional encoding for each time-step. 
    
    Adding positional encoding enriches the information fed into the Generator. It enables the Generator to take into account the spatial 
    connectivity of the beats. Similar to processing a string of words, the ordering of the words is probably the most important information in
    determining the meaning of the sentence-so is the ordering of the notes in a song-the team is hoping that the positional encoding can provide
    this ordering information to the Generator. The advantage of this adopted positional encoding lies in: 1) its ability to provide unique coding
    for the note at each time-step. 2) its ability to generalize to input of arbitrary length. 3) the encoding does not explode as the input length
    increases. 

\end{par}