\begin{par}
    \par \hspace{15pt} Machine learning has been employed extensively in the field of computer vision and image processing, however,it has not been extensively investigated in the music field. For this project, the group aims to use various machine learning models to do music genre classification and music composition. More particularly, the group intends to produce a drum track generator model, which takes the the rest of the song (song tracks without the drum track) and the predicted genre as inputs, then outputs a drum track sequentially based on those inputs.
\end{par}

\begin{par}
    \par \hspace{15pt} Compared to copious deep learning researches in vision, acoustic generative networks are less visited. The group believes that it would be a lucrative field since amateur composers may not be able to compose for all instruments, which means most of them are more familiar with some instruments than the others. In that case, our model, with the ability to generate a music track (drum track) based on tracks from other instruments, can provide important insights for those amateur composers with their music compositions. Moreover, since the aim of the group is to provide a computationally-light training model, it can be easily re-trained towards different musical instruments, even novel ones such as theremin or Kazoo with contemporary styles. 
\end{par}